\documentclass[]{llncs}
\usepackage{amsmath}
\usepackage{amssymb}
\usepackage{mathtools}
\usepackage{proof}
\usepackage{graphicx}
\usepackage{svg}

\title{Towards an authorization framework for app security checking}
\author{Joseph Hallett
  \and David Aspinall
  %\and Nassim Seghir
  %\and Wei Chen
  \thanks{With thanks to Martin Hofmann and Andy Gordon for their time and advice.}
}
\institute{University of Edinburgh}
\date{\today}

% A) Position papers of PhD students in an early stage of their project should
%    fulfill the following requirements.
% * Length: two to six pages
% * Format: Submissions should be formatted according to the LNCS guidelines.
% * Content:
%   * Authors’ names (PhD student + contributing team members if applicable)
%   * affiliation
%   * an abstract (maximum 200 words)
%   * the problem(s) that the proposed research is going to solve, and the
%     motivation for solving them
%   * the aims and objectives of the proposed research
%   * the potential contributions to the state-of-the-art
%   * the research methodology to be used to achieve the research goals,
%     including a brief description of the work done to date and a tentative
%     research plan for future work
%   * the main contribution(s) of the research to the field of Engineering
%     Secure Software and Systems

\newcommand{\keyword}[1]{\textsf{#1}}
\newcommand{\entity}[1]{\textrm{#1}}
\newcommand{\vari}[1]{\textsl{#1}}

\newcommand{\says}{\keyword{~says~}}
\newcommand{\cansay}[1]{\keyword{~can say}_{#1}~}
\newcommand{\iffacts}{\keyword{~if~}}
\newcommand{\meets}{\keyword{~meets~}}
\newcommand{\shows}{\keyword{~shows~}}
\newcommand{\installable}{\keyword{~is installable}}

\newcommand{\Phone}{\entity{Phone}}
\newcommand{\Nll}{\entity{NoInfoLeaks}}
\newcommand{\Nlli}{\entity{NILInferer}}
\newcommand{\Nm}{\entity{NotMalware}}
\newcommand{\Avc}{\entity{AVChecker}}
\newcommand{\Google}{\entity{Google}}
\newcommand{\App}{\entity{Game}}
\newcommand{\Evi}{\entity{Evidence}}

\newcommand{\LocalChecks}[2]{\textsf{Local#2Check}{\left(#1\right)}}

\newcommand{\app}{\vari{app}}
\newcommand{\policy}{\vari{policy}}
\newcommand{\evi}{\vari{evidence}}
\newcommand{\fact}{\vari{fact}}

\newcommand{\rrule}[1]{\textsc{\small #1}}
\newcommand{\rcansay}{\rrule{can say}}
\newcommand{\rcond}[1]{\rrule{cond #1}}

\newcommand{\eref}[1]{(\ref{#1})}

\begin{document}
\maketitle

\begin{abstract} 
  
  Apps don't come with any guarantees that they are not malicious.  This paper
  introduces a PhD project designing the authorization framework used for App
  Guarden. App~Guarden is a new project that uses a flexible assurance
  framework based on distribution of evidence, attestation and checking
  algorithms to make explicit why an app isn't dangerous and to allow users to
  describe how they want apps on their devices to behave.  We use the SecPAL
  policy language to implement a device policy and give a brief example of a
  policy being used. Finally we use SecPAL to describe some of the differences
  between current app markets. 

\end{abstract}

\section{Introduction}

App~stores allow users to buy and install software on their devices with a
minimum of fuss.  Users trust the app stores not to supply them with malware
but sometimes this trust is misplaced. A survey of the various different
Android markets revealed some malware in all of them---including Google's own
Play Store~\cite{Zhou:2012tb}.  Malware on mobile platforms typically steals
user information and monetises itself by sending premium rate text
messages~\cite{Felt:2011we}.  
Sometimes malware on mobile platforms isn't intentional: developers are not
security specialists and sometimes make poor security decisions such as signing
apps with the same key~\cite{Barrera:2012ib} which can lead to permission
escalation on Android or by requesting excessive
permissions~\cite{Felt:2011kj}.

Both Google and Apple (who operate the two biggest app markets) check apps
submitted to them for malicious code, however neither states exactly what they
check for. There is expectation amongst users that these markets check for and
are free from malware, but this trust is misplaced~\cite{Enck:2010ta}. Attempts
to reverse engineer the checking procedures for Google's store suggest that
they incorporate static and dynamic analysis as well as manual checking by a
human being~\cite{Oberheide:2012tl}. They offer no guarantees however that any
check was actually run.

Digital evidence~\cite{Stark:2009uc} can be used to confirm the results from
static analyses; it is similar to older techniques~\cite{Necula:1996tr} and
allows us to split the \emph{inferring} of a static analysis result (which may
be computationally expensive) from the \emph{checking} the evidence (which
should be efficient).  This allows us to verify the results of static analyses
for security properties without having to fully run the analysis.

Static analysis checks used to create digital evidence could include taint
analysis tools to check where sensitive information is being leaked between
apps~\cite{Enck:2010uw}, or a security proof based on the architecture of the
machine~\cite{Barthe:2007bd}.  

The App~Guarden project aims to use digital evidence to provide apps with
guarantees that the app cannot act maliciously.  This will allow us to make
explicit the checks made on an app.  With explicit checks users will be able to
write policies which describe what kinds of apps they will allow on their
devices, and what level of trust they require in these checks for an app to be
considered \emph{installable}.  We aim to create an enhanced app~store that
uses inference services, and assurance logics on a modified Android to increase
trust in devices. 

Figure~\ref{fig:architecture} shows an App~Guarden store. It provides apps with
evidence generated by an inference service.  Devices can check the evidence
using a trusted checking service, or rely on their trust that the store
believed the app to be safe.  The SecPAL Engine on the phone is used to check
the assertions made by the store and the checker service and enforce a device
policy.

\begin{figure}[h]
  \centering
  \def\svgwidth{0.57\textwidth}
  \scriptsize\sf
  \input{architecture.eps_tex}
  \caption{The architecture of App~Guarden.}\label{fig:architecture}
\end{figure}

\section{Aims and Objectives}

Our aim is to create a modified Android ecosystem where apps
come with digital evidence for security and where users can say how
they want apps to behave on their device. To do this we will:
\begin{itemize}
  \item Show how an authorization logic can specify a policy for how a
    device behaves.  This will use ideas
    such as proof checking and inference, and show how trust can be
    distributed between principals.

  \item Show how to use the logic to specify a device policy. We will use
    digital evidence, and assertions between trusted principals and show how
    the trust relationships can be propagated between devices. We show how we
    can use these statements to build confidence that an app is secure.

  \item Describe the differences between current app markets using the
    logic and show how they manage access to special resources
    (such as the address book or a user's photos).  This will allow us to
    compare App~Guarden to other systems and allow comparison between other
    different schemes.

  \item Put the logic in a real device and app store and see how
    the system behaves with real malware and how they interact with device
    policies.  We will look for ways to attack devices with our framework and
    explore it's limitations.
\end{itemize}


\section{SecPAL and App~Guarden}

Suppose a user has a mobile device.  They might wish that:

\begin{quote}
  ``No app installed on my phone will send my location to an advertiser, and I
  wont install anything that Google says is malware.''
\end{quote}

We call this their \emph{device policy}.  It says what the user will allow on
their device for an app to be installable. To formally write the device
policies we use the SecPAL authorization language~\cite{Becker:2006vh}. The
device policy is show in~\eref{sp:DevicePolicy}. SecPAL is designed for
distributed access control decisions.  It was designed to be human readable
which makes it ideal for our application as we would like users to be able to
understand (if perhaps not write) their own device policies. Another advantage
is that it lets us separate the constraint solving (the digital evidence
checking) from the authorization logic.  This means there is no restriction on
having the digital evidence be in the same format and logic as the assurance
framework (as is the case with other authorization logics such as
BLF~\cite{Whitehead:2004bu}); this increases our flexibility for implementation
as we can re-purpose existing logic solvers.

SecPAL is designed to be extensible.  Authorization statements take the form of
assertions made by principals.  Statements can include predicates which are
defined by the SecPAL program, as well as \emph{constraints} (as
seen in~\eref{sp:NlliSaysEviShowsAppPol}) that can be checked as sub-procedures
for the main query engine.  We add the following statements: 

\begin{align}
  e_\text{evidence}\shows e_\text{app}\meets e_\text{policy} \label{eqn:shows}\\
  e_\text{app}\meets e_\text{policy} \label{eqn:meets}
\end{align}

The\shows\dots\meets{}predicate in~\eref{eqn:shows} tells us that some evidence
could be checked to show that an app satisfies a security policy; an example of
which can be seen in~\eref{sp:NlliSaysEviShowsAppPol} bellow (and which also
uses a constraint). The plain\meets{}in~\eref{eqn:meets} is an assertion that
an app meets a policy whilst offering no actual evidence that the app is
secure. The\shows{}statement is stronger than the\meets{}statement as anyone
who says that some evidence shows an app meets a policy suggests they would
also say the app meets policy.  We can write this in SecPAL with the assertion:
``\(\Phone\says\app\meets\vari{policy}\iffacts\vari{evidence}\shows\app\meets\policy\).''

We also add an\installable~statement to indicate an app is installable on a
device and a \keyword{requires} statement to say an app needs a permission to
run.

SecPAL allows for attestation of said statements with the \emph{can say}
statement: these come with a delegation depth.  A delegation depth of zero (as
shown in~\eref{sp:NlliCanSayNll}) means the statement must be spoken by the
principal and given to us directly
in the assertion context. A delegation depth of infinity (as
in~\eref{sp:GoogleCanSayNm}) means that the statement might not be given to us
directly but rather inferred from a statement by different principal but with
several  \emph{can say} phrases linking it back to the required principal.


%\section{Querying SecPAL}

To evaluate the device policy and decide whether a specific app is installable
SecPAL requires a set of assertions called the \emph{assertion-context}. An
example assertion-context and the proof, using two of SecPAL's evaluation rules
from the original paper~\cite{Becker:2006vh}. To illustrate a usage of SecPAL
we give an example where a user wishes to decide whether an app \App{} can be
installed. To do this we use the assertion context bellow:

\begin{align}
    AC & \coloneqq \left\{ \right. \nonumber\\
    \Phone&\says \app\installable \nonumber\\
    \iffacts&\app\meets\Nm, \nonumber\\
            &\app\meets\Nll.
  \label{sp:DevicePolicy}\\
    \Phone&\says \app\meets\policy \iffacts \evi\shows\app\meets\policy.
  \label{sp:showsmeets}\\
    \Phone&\says\Nlli\cansay 0\app\meets\Nll.
  \label{sp:NlliCanSayNll}\\
    \Phone&\says
    \Google\cansay\infty
    \app\meets\Nm.
  \label{sp:GoogleCanSayNm}\\
    \Google&\says\Avc\cansay 0\app\meets\Nm.
  \label{sp:AvcCanSayNm}\\
    \Avc&\says\App\meets\Nm.
  \label{sp:AvcSaysNm}\\
    \Nlli&\says\Evi\shows\App\meets\Nll \nonumber\\
         & \iffacts \LocalChecks{\Evi,  \App}{NIL}=\textsf{true}.
  \label{sp:NlliSaysEviShowsAppPol}\\
  \left.\right\}\nonumber
\end{align}

We evaluate it using the SecPAL rules shown in the
\textsc{cond}~\eref{rule:cond} and \textsc{can say}~\eref{rule:cansay} rules.
The \textsc{cond} rule is SecPAL's equivalent to the \emph{modus ponens} rule.
Whereas \textsc{can say} is similar to \emph{speaks for} relationships
\cite{Abadi:2003kta}:  if a first principal says a second principal can say
something and the second does say it then the first says it too. $AC$
represents the assertion context, $D$ is the current delegation level. From
these rules and the assertion context we can prove an assertion that
``\(\Phone\says\App\installable\).''

\begin{equation}
  \scalebox{0.8}{%
  \infer[\rcond{}]{%
    \left\{A\says\fact\iffacts\fact_1,  \dots\fact_n,  c\right\}\cup AC,  D \models
      A\says\fact
  }{%
    \begin{split}
    \left\{A\says\fact\iffacts\fact_1,  \dots,  \fact_n,  c\right\}\cup AC,  D \models
      A\says\fact_1,  \dots,  \fact_n \\
    \models c \\
    vars(\fact) = \emptyset
    \end{split}
  }}
  \label{rule:cond}\\
\end{equation}

\begin{equation}
  \scalebox{0.8}{%
  \infer[\rcansay]{%
    AC,  \infty\models A\says\fact
  }{%
    AC,  \infty\models A\says B\cansay D\fact &
    AC,  D\models B\says\fact
  }}
  \label{rule:cansay}
\end{equation}

\section{Comparing different devices}

We can use the SecPAL language to describe the differences between various
app markets and devices.  The two most popular operating systems are Google's
Android, and Apple's {iOS}. Apple's offering is usually considered to be the more
restrictive system as it doesn't allow the use of alternate market places,
whereas an Android user can choose to relax the restrictions and install from
anywhere.

For example on iOS an app is only installable on an iPhone if Apple has said
the app can be installed~\eref{oas:iphoneinstall}. In contrast on Android any
app can be installed if the user says so~\eref{oas:anduser}. The user also
trusts any app available on Google's Play Store~\eref{oas:androidinstall}, but
the zero delegation-depth means that the user will only believe the app is
installable if Google provides the assertion directly whereas the user
in~\eref{oas:anduser} is free to delegate the decision because of the infinite
delegation depth.

\begin{align}
  \entity{iPhone}&\says\entity{Apple}\cansay\infty\app\installable.
  \label{oas:iphoneinstall}\\
  \\
  \entity{AndroidPhone}&\says\entity{AndroidUser}\cansay\infty\app\installable.\label{oas:anduser}\\
  \entity{AndroidUser}&\says\Google\cansay0\app\installable.
  \label{oas:androidinstall}
\end{align}

Similarly comparisons can be made when apps try to access resources.
When accessing components like the address book or photos; on iOS this
is only allowed when the user has explicitly okayed it. Other operating systems
such as Windows~Mobile also follow this pattern.
On Android, however, an app can access any resource it declared itself as
\emph{requiring} it when it was installed.

\begin{align}
  \entity{iPhone}&\says\entity{User}\cansay0\app\keyword{~can access~}\entity{resource}.\\
  \\
  \entity{AndroidPhone}&\says\app\keyword{~can access~}\entity{resource} \nonumber \\
  \iffacts&\app\installable, \nonumber\\
  &\app\keyword{~requires~}\entity{resource}.
\end{align}

This is a brief comparison of the implicit device policies in current markets.
It shows however that SecPAL is capable of describing the various difference.
This could be extended in future to allow proper comparisons of different
markets.

\section{Conclusion and future directions}

We have given a brief introduction to App Guarden and the SecPAL language used
to describe policies.
Using the SecPAL language we described some of the differences between current
systems. The PhD work here is at an early stage. Next steps will include
porting the SecPAL language to Android and using it to find what kinds of
device policies are most effective at stopping dangerous apps, without overly
inconveniencing the user. From here we will build an \emph{enhanced app market}
with its own set of policies where apps come with
digital evidence to allow users to trust apps further.

Mobile devices are a growing target for malware, and with app stores appearing
in conventional computers the differences between mobile and traditional
computing environments are blurring.  The App Guarden will help give users
greater assurance their software is secure; and provide a new security
mechanism to compliment access control and anti-malware techniques.



\appendix

%\section{Inference tree}
%
%\begin{align}
%  \scalebox{0.8}{%
%  \infer[\rcansay]{%
%    AC,  \infty\models\Phone\says\App\meets\Nm
%  }{%
%    AC,  \infty\models\eref{sp:GoogleCanSayNm} &
%    \infer[\rcansay]{%
%      AC,  \infty\models\Google\says\App\meets\Nm
%    }{%
%      AC,  \infty\models\eref{sp:AvcCanSayNm} &
%      AC,  0\models\eref{sp:AvcSaysNm}
%    }
%  }
%  }\label{prf:nm}\\
%  \nonumber\\
%  \scalebox{0.8}{%
%  \infer[\rcansay]{%
%    AC,  \infty\models\Phone\says\App\meets\Nll
%  }{%
%    AC,  \infty\models\eref{sp:NlliCanSayNll} &
%    \infer[\rcond{\ref{sp:showsmeets}}]{%
%      AC,  0\models\Nlli\says\App\meets\Nll
%    }{%
%      \infer[\rcond{\ref{sp:NlliSaysEviShowsAppPol}}]{%
%        AC,  0\models\Nlli\says\Evi\shows\App\meets\Nll
%      }{%
%        \models\LocalChecks{\Evi,  \App}{NIL}=\textsf{true}
%      }
%    }
%  }}
%  \label{prf:nil}\\
%  \nonumber\\
%  \scalebox{0.8}{%
%  \infer[\rcond{\ref{sp:DevicePolicy}}]{%
%    AC,  \infty\models\Phone\says\App\installable
%  }{%
%    \eref{prf:nm} & \eref{prf:nil}
%  }}
%  \label{prf:dp}
%\end{align}


\bibliographystyle{plain}
\bibliography{appguarden}
\end{document}

